% Author: Cameron F. Abrams, <cfa22@drexel.edu>
%
% To use this template, create a yaml file containing a list
% of items, each of which needs the following key:value pairs:
%
%  Q: the question text, with ordered blank(s) represented by ___ (3 underscores)
%  A: the answer(s), same order as blank(s) (as a string)
%  text: an explanation to be included in the solutions
%  
% refer to this in as the 'config' in the exam description yaml file.  If you
% are building a series of exams in a subdirectory, prepend the name with 
% '../' 
\begin{pycode}
import numpy as np
import yaml
config = '<<<config>>>'
serial = <<<serial>>>
qno = <<<qno>>>
group = <<<group>>>
rng = np.random.default_rng(seed=serial)
with open(config,'r') as f:
    SAQ = yaml.safe_load(f)
instructions = 'Short answer questions. Provide your answer in the space below each question.'
if 'content' in SAQ and 'config' in SAQ:
    c = SAQ['config']
    SA = SAQ['content']
    shuffle = c.get('shuffle', True) if 'c' in locals() else True
    if shuffle: 
        rng.shuffle(SA)
    count = c.get('count',len(SA))
    SA = SA[:count]
    instructions = c.get('instructions', instructions)
else:
    SA = SAQ
    shuffle = c.get('shuffle', True) if 'c' in locals() else True
    if shuffle: 
        rng.shuffle(SA)
answers = [x['A'] for x in SA]
if 'AnsSet' in locals():
    for l, t in zip([chr(ord('a')+x) for x in range(len(answers))], answers):
        AnsSet.register(qno, group=group, label=l, value=t)
else:
    soln=f'{serial}_{qno}_soln.yaml'
    with open(soln,'w') as f:
        yaml.dump(dict(source='short_answer.tex', serial=serial, answers=answers), f)
\end{pycode}
\py{instructions}
\begin{pycode}
print(r'\begin{enumerate}'+'\n')
idx_char = 'a'
for idx, qdict in enumerate(SA):
    idx_char = chr(ord(idx_char)+idx)
    print(r'\item[' + idx_char + '.] ' + qdict['Q'] + '\n')
    print(r'\ifshowsolutions\textcolor{blue}{' + str(qdict['A']).strip() + r'}' + '\n')
    print(r'\textcolor{red}{'+qdict['text']+r'}'+'\n')
    print(r'\else\vspace{1.4cm}\fi'+'\n')
print(r'\end{enumerate}'+'\n')
if 'clearpage' in c and c['clearpage']:
    print(r'\clearpage'+'\n')
\end{pycode}

