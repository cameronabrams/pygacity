% Author: Cameron F. Abrams, <cfa22@drexel.edu>
%
% To use this template, create a yaml file containing a list
% of items, each of which needs the following key:value pairs:
%
%  Q: the question text
%  choices: a dict of choice_label: choice_text pairs
%  A: the answer (the choice_label corresponding to the correct choice)
%  text: an explanation to be included in the solutions
%  
% refer to this in as the 'config' in the exam description yaml file.  If you
% are building a series of exams in a subdirectory, prepend the name with 
% '../' 
\begin{pycode}
import numpy as np
import yaml
config = '<<<config>>>'
serial = <<<serial>>>
qno = <<<qno>>>
group = <<<group>>>
rng = np.random.default_rng(seed=serial)
with open(config,'r') as f:
    MCQ = yaml.safe_load(f)
if 'content' in MCQ and 'config' in MCQ:
    c = MCQ['config']
    MC = MCQ['content']
    shuffle = c.get('shuffle', True) if 'c' in locals() else True
    if shuffle: 
        rng.shuffle(MC)
    count = c.get('count',len(MC))
    MC = MC[:count]
else:
    MC = MCQ
    shuffle = c.get('shuffle', True) if 'c' in locals() else True
    if shuffle: 
        rng.shuffle(MC)
shuffle_choices = c.get('shuffle_choices', True) if 'c' in locals() else True
if shuffle_choices:
    for qdict in MC:
        correct_choice_text = qdict['choices'][qdict['A']]
        all_choice_texts = list(qdict['choices'].values())
        rng.shuffle(all_choice_texts)
        qdict['choices'] = dict(zip(qdict['choices'].keys(), all_choice_texts))
        for clabel, ctext in qdict['choices'].items():
            if ctext == correct_choice_text:
                qdict['A'] = clabel
answers = [x['A'] for x in MC]
if 'AnsSet' in locals():
    for l, t in zip([chr(ord('a')+x) for x in range(len(answers))], answers):
        AnsSet.register(qno, group=group, label=l, value=t)
else:
    soln=f'{serial}_{qno}_soln.yaml'
    with open(soln,'w') as f:
        yaml.dump(dict(source='multiple_choice.tex', serial=serial, answers=answers),f)
\end{pycode}
Multiple Choice questions. Circle the letter of the correct answer from the choices provided.
\begin{pycode}
print(r'\begin{itemize}'+'\n')
for i, qdict in enumerate(MC):
    print(r'\item['+chr(ord('a')+i)+r'.] '+qdict['Q']+'\n')
    print(r'\begin{itemize}'+'\n')
    for clabel, ctext in qdict['choices'].items():
        if clabel == qdict['A']:
            print(r'\item[\ifshowsolutions\textcolor{blue}{\fbox{\textbf{'+clabel+r'}}}\else '+clabel+r'\fi.] '+str(ctext)+'\n')
        else:
            print(r'\item['+clabel+r'.] '+str(ctext)+'\n')
    print(r'\end{itemize}'+'\n')
    print(r'\ifshowsolutions\textcolor{red}{'+qdict['text']+r'}\fi'+'\n')
print(r'\end{itemize}'+'\n')
if 'clearpage' in c and c['clearpage']:
    print(r'\clearpage'+'\n')
\end{pycode}

