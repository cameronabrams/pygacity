% Automatically generated LaTeX source file
\documentclass[11pt, solutions]{autoprob}
\usepackage[scaled]{helvet}
\renewcommand\familydefault{\sfdefault}

\renewcommand{\Universityname}{University of Nowhere}
\renewcommand{\Departmentname}{ABC 1234}
\renewcommand{\Coursename}{BIO 5678 - Introduction to Applied Pyrohydrodynamics}
\renewcommand{\Termname}{Winter 2025-2026}
\renewcommand{\Termcode}{202525}
\renewcommand{\Instructorname}{Eustace T. Smartypants}
\renewcommand{\Instructoremail}{ets@unow.edu}
\pagestyle{fancy}
\fancyhf{}
\lhead{\footnotesize \Universityname\ --- \Coursename\ --- \Termname}
\rhead{\footnotesize 68549586}
\rfoot{\thepage}
\begin{document}
\begin{pycode}
import yaml
import random
import numpy as np
import fractions as fr
import logging
import pickle
from argparse import Namespace
from pathlib import Path

import pygacity.util.texutils as tu
from pygacity.generate.answerset import AnswerSet
from pygacity.generate.pick import Picker
from pygacity.util.collectors import FileCollector

serial = 68549586
rng = np.random.default_rng(seed=serial)
Pick = Picker(serial)
AnsSet = AnswerSet(serial)
loglevel_numeric = getattr(logging, 'DEBUG')
logging.basicConfig(filename=f'pythontex-{serial}.log',
                    filemode='w',
                    format='%(asctime)s %(name)s %(message)s',
                    level=loglevel_numeric)
logger = logging.getLogger(__name__)
pythonTexFC = FileCollector()
pickle_cache = Path.cwd() / ".cache"
# pickle_cache should already exist if the manager makes it


import os
from pathlib import Path
safe_mplconfig = Path.cwd() / "_mplconfig"
safe_mplconfig.mkdir(exist_ok=True)
os.environ["MPLCONFIGDIR"] = str(safe_mplconfig)

import matplotlib
matplotlib.use("Agg")
import matplotlib.pyplot as plt

logging.getLogger("matplotlib").setLevel(logging.WARNING)



from sandlersteam.state import State as SANDLER
from sandlersteam.state import SteamTables as st
from sandlersteam.request import Request
STReq = Request()
suphPavail = st['suph'].uniqs['P']


\end{pycode}

\examheader{Exam I}{January 13, 2026}
\noindent\textbf{Instructions:} This exam is closed book, and you are allowed 
3 pages of your own handwritten notes on paper.  No electronic devices except 
for a non-smartphone, non-tablet based calculator. Show all work for full credit. 
Clearly indicate your final answers. The total number of points is 100. You have 
110 minutes to complete the exam.


\begin{enumerate}
\item \begin{pycode}
idx = 4.1
group = 1
MIXER = Pick.pick_state(
    dict(
        P1={'default':14,'pick':{'between':[13,15],'round':2}},
        P2={'default':12,'pick':{'between':[10,12.5],'round':2}},
        T1={'default':1000,'pick':{'pickfrom':[900,950,1000,1050,1100]}},
        T2={'default':400,'pick':{'pickfrom':[350,375,400,425,450]}},
        T3={'default':600,'pick':{'pickfrom':[500,550,600,650,700]}},
        CP_over_R={'default':3.5,'pick':{'pickfrom':[7,7.5,8]}}
    )
)
P1 = MIXER.P1
P2 = MIXER.P2
T1 = MIXER.T1
T2 = MIXER.T2
T3 = MIXER.T3
CP_over_R = MIXER.CP_over_R
alpha = (T1 - T3) / (T3 - T2)
P3 = pow(pow(T3**(1 + alpha) / T1 / T2**alpha, CP_over_R) * P1 * P2**alpha, 1 / (1 + alpha))
cpr = tu.frac_or_int_as_tex(fr.Fraction(CP_over_R))
if 'AnsSet' in locals():
    AnsSet.register(idx, group=group, label=r'\(\alpha\)', value=np.round(alpha,5), formatter='{:.5f}')
    AnsSet.register(idx, group=group, label=r'\(P\)', value=np.round(P3,3), units='bar', formatter='{:.3f}')
\end{pycode}

A stream of air at \py{P1} bar and \py{T1} K (labeled ``stream 1'') is to be cooled to \py{T3} K by mixing with another stream of air at \py{P2} bar and \py{T2} K (labeled ``stream 2'').  Let $\alpha$ be the ratio of the molar flow rate of the hotter stream to that of the cooler stream.  Compute (1) $\alpha$, and (2) the pressure $P$ of the mixed stream (labeled ``stream 3'').  You may assume this is carried out adiabatically and that air is an ideal gas for which $C_{\rm P}$ = $\py{cpr}R$.

It may be helpful for you to remember, {\bf for the ideal gas}, that a change of state from $(T_A,P_A)$ to $(T_B,P_B)$ results in the following enthalpy and entropy changes, respectively:

\begin{align*}
\Delta \molar{H} \equiv \molar{H}_B-\molar{H}_A = \int_{T_A}^{T_B} C_{\rm P} dT\\
\Delta \molar{S} \equiv \molar{S}_B-\molar{S}_A = \int_{T_A}^{T_B} \frac{C_{\rm P}}{T} dT - R \ln\frac{P_B}{P_A}.
\end{align*}

\ifshowsolutions\Solutionheader

Let $\dot{n}$ be the unknown molar flow rate of stream 1.  This means the outlet stream (3) has a flow rate of $\alpha\dot{n}$.  An energy balance here resolves to 
\begin{align*}
H_{\rm out} & = H_{\rm in}\\
(1+\alpha) \dot{n} \Hb(T_3,P_3) & = \dot{n}\Hb(T_1,P_1) + \alpha\dot{n} \Hb(T_2,P_2)\\
(1+\alpha)  \int_{T_r}^{T_3} C_{\rm P} dT & = \int_{T_r}^{T_1} C_{\rm P}dT + \alpha\int_{T_r}^{T_2} C_{\rm P}dT\\
(1+\alpha)C_{\rm P}(T_3-T_r) & = C_{\rm P}(T_1-T_r) + \alpha C_{\rm P}(T_2-T_r)\\
(1+\alpha)T_3 & = T_1 + \alpha T_2\\
\Rightarrow T_3 & = \frac{T_1+\alpha T_2}{1+\alpha},\ \mbox{or}\\
\alpha & = \frac{T_1-T_3}{T_3-T_2}\\
 & = \frac{900-500}{500-400} = \frac{400}{100} = \fbox{\py{f'{alpha:.1f}'}}.
\end{align*}
(Note all terms involving $T_r$ cancel and $C_{\rm P}$ divides out.).  We can get $P_3$ from an entropy balance:
\begin{align*}
S_{\rm out} & = S_{\rm in}\\
(1+\alpha)\dot{n} \sb(T_3,P_3) & = \dot{n}\sb(T_1,P_1) + \alpha\dot{n} \sb(T_2,P_2)\\
\sb(T_3,P_3) - \sb(T_1,P_1) + \alpha\left[\sb(T_3,P_3)-\sb(T_2,P_2)\right] & = 0\\
C_{\rm P}\ln\frac{T_3}{T_1} - R\ln\frac{P_3}{P_1} + \alpha\left[C_{\rm P}\ln\frac{T_3}{T_2}-R\ln\frac{P_3}{P_2}\right] & = 0\\
C_{\rm P}\ln\left(\frac{T_3^{1+\alpha}}{T_1T_2^\alpha}\right) - R\ln\left(\frac{P_3^{1+\alpha}}{P_1P_2^\alpha}\right) & = 0\\
\ln\left[\left(\frac{T_3^{1+\alpha}}{T_1T_2^\alpha}\right)^{\frac{C_{\rm P}}{R}}\right] & = \ln\left(\frac{P_3^{1+\alpha}}{P_1P_2^\alpha}\right)\\
\Rightarrow\ P_3 & = \left[\left(\frac{T_3^{1+\alpha}}{T_1T_2^\alpha}\right)^{\frac{C_{\rm P}}{R}}P_1P_2^\alpha\right]^{\frac{1}{1+\alpha}}\\
& = \left[\left(\frac{(\py{T3})^{(1+\py{f'{alpha:.2f}'})}}{(\py{T1})(\py{T2})^{\py{f'{alpha:.2f}'}}}\right)^{\py{cpr}}(\py{P1})(\py{P2})^{\py{f'{alpha:.2f}'}}\right]^{\frac{1}{1+\py{f'{alpha:.2f}'}}} \\
& = \fbox{\py{f'{P3:.2f}'}\ \mbox{bar}}.
\end{align*}
\clearpage
\else
\ifgivespace
\clearpage
\noindent (This page left blank for extra space for Problem \theenumi.)
\clearpage
\fi
\fi

\item \begin{pycode}
idx = 4.2
group = 1
DSUH=Pick.pick_state(
    dict(
        P1={'default':3.2,'pick':{'between':[2.9,3.9],'round':1}}, # MPa
        T1C={'default':355,'pick':{'between':[340,360],'round':0}},
        Psat={'default':2.9,'pick':{'between':[2.7,2.9],'round':3}}, # MPa
        TLC={'default':50,'pick':{'between':[45,55],'round':0}},
        mdot={'default':15,'pick':{'between':[10,50],'round':0}}))
P1  = DSUH.P1
T1C = DSUH.T1C
Psat = DSUH.Psat
TLC = DSUH.TLC
mdot = DSUH.mdot
inlet = SANDLER(T=T1C, P=P1)
outlet = SANDLER(P=Psat, x=1.0)
inL = SANDLER(T=TLC, x=0.0)

mLdot = mdot * ((inlet.h - outlet.h)/(inlet.h - inL.h))
m_show = float(np.round(mLdot, 3))
if 'AnsSet' in locals():
    AnsSet.register(idx, group=group, label=r'\(\dot{m}_\mathrm{L}\)', value=m_show, units='kg s$^{-1}$', formatter='{:.3f}')
\end{pycode}
Superheated steam at \py{f'{P1:,.0f}'} MPa and \py{T1C}$^\circ$C is to be converted to saturated steam at \py{f'{Psat:,.0f}'} MPa in a desuperheater.  This desuperheater is supplied with inlet liquid water at \py{TLC}$^\circ$C.  The unit should produce saturated steam at a rate of \py{mdot} kg~s$^{-1}$.  Assuming adiabatic operation, and assuming the liquid inlet is saturated, what is the mass flowrate of the inlet water?

The following enthalpies will be useful:
\begin{itemize}
\item[] Superheated steam at \py{T1C}$^\circ$C and \py{f'{P1:,.0f}'}~MPa: $\hat{H}$ = \py{f'{inlet.h:,.2f}'} kJ/kg;
\item[] Saturated liquid water at \py{TLC}$^\circ$C: $\hat{H}^{\rm L}$ = \py{f'{inL.h:,.2f}'} kJ/kg; and
\item[] Saturated water vapor at \py{f'{Psat:,.0f}'}~MPa: $\hat{H}^{\rm V}$ = \py{f'{outlet.h:,.2f}'} kJ/kg.
\end{itemize}

\ifshowsolutions\Solutionheader

Let stream 1 be the liquid water stream, which we assume is saturated liquid, stream 2 be the superheated steam inlet, and stream 3 be the saturated steam outlet.  Hence, $\dot{m}_3$ = 15 kg~s$^{-1}$ as given.  The mass and energy balance yield the two unknowns $\dot{m}_1$ and $\dot{m}_2$:

\begin{align*}
\dot{m}_1 + \dot{m}_2 & = \dot{m}_3\\
\dot{m}_1\hat{H}_1 + \dot{m}_2\hat{H}_2 & = \dot{m}_3\hat{H}_3\\
\end{align*}

Solving these two simultaneously yields 

\begin{align*}
\dot{m}_1 & = \dot{m}_3\left(\frac{\hat{H}_2-\hat{H}_3}{\hat{H}_2-\hat{H}_1}\right)\\
& = (15)\left(\frac{\py{f'{inlet.h:,.2f}'}-\py{f'{outlet.h:,.2f}'}}{\py{f'{inlet.h:,.2f}'}-\py{f'{inL.h:,.2f}'}}\right) = \fbox{\py{f'{mLdot:.2f}'}\ \mbox{kg~s$^{-1}$}}.
\end{align*}
\clearpage
\else
\ifgivespace
\clearpage
\fi
\fi


\item \begin{pycode}
from scipy.optimize import fsolve
import pandas as pd

idx = 4.3
group = 1

def acm_ocm(x, A):
    return np.array([np.exp(A * (1 - x)**2), np.exp(A * x**2)])

def bubp(x, Pvap=[], acm=None, acmp=[]):
    g = acm(x, *acmp)
    Ptot = x * g[0] * Pvap[0] + (1 - x) * g[1] * Pvap[1]
    yres = x * g[0] * Pvap[0] / Ptot
    return Ptot, yres

def dewp(y, Pvap=[], acm=None, acmp=[], xinit=0.5):
    def f_dewp(xguess, ytarg, Pvap, acm, acmp):
        P, ycomp = bubp(xguess, Pvap, acm, acmp)
        return ytarg - ycomp
    result = fsolve(f_dewp, xinit, args=(y, Pvap, acm, acmp))
    x = result[0]
    P, ydum = bubp(x, Pvap, acm, acmp)
    return P, x

T = 343.15
Pv = np.array([79.8, 40.5])
Aij = 1.15
xbub = 0.08
g = acm_ocm(xbub, Aij)
Pbub, ybub = bubp(xbub, Pvap=Pv, acm=acm_ocm, acmp=[Aij])

ydew = 0.12
xinit = 0.02
Pdew, xdew = dewp(ydew, Pvap=Pv, acm=acm_ocm, acmp=[Aij], xinit=xinit)
gdew = acm_ocm(xdew, Aij)
Pcheck, ycheck = bubp(xdew, Pvap=Pv, acm=acm_ocm, acmp=[Aij])

# newton raphson for dew point
def nr_f(x,y,Pvap=[],acm=None,acmp=[]):
    g = acm(x, *acmp)
    return y * (1 - x) * g[1] * Pvap[1] - (1 - y) * x * g[0] * Pvap[0]

def nr_dfdx(x,y,Pvap=[],acm=None,acmp=[]):
    g = acm(x, *acmp)
    A = acmp[0]
    return -(1 - 2 * A * x * (1 - x)) * (y * g[1] * Pvap[1] + (1 - y) * g[0] * Pvap[0])

xsav = [xinit]
x = xsav[-1]
g = acm_ocm(x, Aij)
g1sav = [g[0]]
g2sav = [g[1]]
dsav = [1 - 2 * Aij * x * (1 - x)]
fsav = [nr_f(x, ydew, Pv, acm_ocm, [Aij])]
dfsav = [nr_dfdx(x, ydew, Pv, acm_ocm, [Aij])]
dx = -fsav[-1] / dfsav[-1]
epsilon = 1.e-6
while abs(dx) > epsilon:
    xsav.append(xsav[-1] + dx)
    x = xsav[-1]
    g = acm_ocm(x, Aij)
    g1sav.append(g[0])
    g2sav.append(g[1])
    dsav.append(1 - 2 * Aij * x * (1 - x))
    fsav.append(nr_f(x, ydew, Pv, acm_ocm, [Aij]))
    dfsav.append(nr_dfdx(x, ydew, Pv, acm_ocm, [Aij]))
    dx = -fsav[-1] / dfsav[-1]
nrdf = pd.DataFrame({r'$x$': xsav, r'$\gamma_1$': g1sav, r'$\gamma_2$': g2sav, r'$[1-2Ax(1-x)]$': dsav, r'$f$': fsav, r'$df/dx$': dfsav})
nrdf_str = nrdf.to_latex(header=True, index=False, float_format="%.5f")
nrdf.to_csv(f'nr_dewp_{serial}_{idx}.csv', header=True, index=False)
pythonTexFC.append(f'nr_dewp_{serial}_{idx}.csv')

xazeo = 0.5 - 0.5 / Aij * np.log(Pv[1] / Pv[0])
gazeo = acm_ocm(xazeo, Aij)
Pazeo = xazeo * gazeo[0] * Pv[0] + (1 - xazeo) * gazeo[1] * Pv[1]
xplot = np.linspace(0, 1, 101)
gplot = acm_ocm(xplot, Aij)
Pplot = xplot * gplot[0] * Pv[0] + (1 - xplot) * gplot[1] * Pv[1]
yplot = xplot * gplot[0] * Pv[0] * np.reciprocal(Pplot)

if 'AnsSet' in locals():
    AnsSet.register(idx, group=group, label='a', value=np.round(Pbub,2), units='kPa', formatter='{:.2f}')
    AnsSet.register(idx, group=group, label='b', value=np.round(Pdew,2), units='kPa', formatter='{:.2f}')
    AnsSet.register(idx, group=group, label='c \(x_{az}\)', value=np.round(xazeo,4), formatter='{:.4f}')
    AnsSet.register(idx, group=group, label='c \(P_{az}\)', value=np.round(Pazeo,2), units='kPa', formatter='{:.2f}')

df = pd.DataFrame({'x': xplot, 'y': yplot, 'g1': gplot[0], 'g2': gplot[1], 'P': Pplot})
df.to_csv(f'pxy-{serial}-{idx}.csv', header=True, index=False)
pythonTexFC.append(f'pxy-{serial}-{idx}.csv')
logger.debug(f'VLE binary OCM Pxy data saved to pxy-{serial}-{idx}.csv')
fig, ax = plt.subplots(1, 1, figsize=(5, 5))
logger.debug('Plotting VLE binary OCM Pxy diagram')
ax.plot(xplot, Pplot, label='bub')
ax.plot(yplot, Pplot, label='dew')
ax.scatter([xazeo], [Pazeo], label='azeotrope', color='red')
ax.set_xlim([0, 1])
ax.set_xlabel(r'$x_1$, $y_1$')
ax.set_ylabel(r'$P$ [kPa]')
ax.legend()
ax.grid(True)
plt.savefig(f'mypxy-{serial}-{idx}.pdf', bbox_inches='tight')
pythonTexFC.append(f'mypxy-{serial}-{idx}.pdf')
logger.debug(f'VLE binary OCM Pxy plot saved to mypxy-{serial}-{idx}.pdf')
plt.close()

\end{pycode}

For the system of ethyl ethanoate(1)/$n$-heptane(2) at \py{f'{T:.2f}'} K,
\begin{itemize}
\item[] $\ln\gamma_1 = \py{f'{Aij:.3f}'} x_2^2$ \ \ \ \ \ \ \ \ \ \ \ $\ln\gamma_2 = \py{f'{Aij:.3f}'} x_1^2$,
\item[] $P_1^{\rm vap}$ = \py{f'{Pv[0]:.2f}'} kPa \ \ \ \ \  $P_2^{\rm vap}$ = \py{f'{Pv[1]:.2f}'} kPa.
\end{itemize}
Assume you can use low-pressure VLE (i.e., ``Modified Raoult's Law''):\\*[3mm]
$y_iP = x_i\gamma_iP_i^{\rm vap}$\\*[3mm]
and answer the following.
\begin{enumerate}
\item[(a)] Compute the bubble-point pressure at $T$ = \py{f'{T:.2f}'} K, $x_1$ = \py{f'{xbub:.2f}'}.
\item[(b)] Compute the dew-point pressure at $T$ = \py{f'{T:.2f}'} K, $y_1$ = \py{f'{ydew:.2f}'}.
\item[(c)] What is the azeotrope composition and pressure at $T$ = \py{f'{T:.2f}'} K?
\end{enumerate}

\ifshowsolutions\Solutionheader

\begin{enumerate}
\item[(a)] Using Modified Raoult's Law, 
\begin{align*}
    P & = x_1 \gamma_1 P_1^{\rm vap} + x_2 \gamma_2 P_2^{\rm vap}\\
     & = x_1 \exp\left[(\py{f'{Aij:.3f}'})(1-x_1)^2\right] P_1^{\rm vap} + (1-x_1) \exp\left[(\py{f'{Aij:.3f}'})x_1^2\right] P_2^{\rm vap}\\
     & = (\py{f'{xbub:.2f}'})(\py{f'{g[0]:.4f}'})(\py{f'{Pv[0]:.2f}'}) + (1-\py{f'{xbub:.2f}'})(\py{f'{g[1]:.4f}'})(\py{f'{Pv[0]:.2f}'}) = \fbox{\py{f'{Pbub:.2f}'}\ \mbox{kPa}}.
\end{align*}

\item[(b)] We can cast the Modified Raoult's law expressions
\begin{align*}
y_1 P & = x_1 \gamma_1 P_1^{\rm vap}\ \mbox{and}\\
P & = x_1 \gamma_1 P_1^{\rm vap} + x_2 \gamma_2 P_2^{\rm vap}
\end{align*}
into a single equation implicit in $x_1$, which for brevity we will refer to as just $x$ (and $y$ instead of $y_1$):
\begin{align*}
y & = \frac{x \gamma_1 P_1^{\rm vap}}{x \gamma_1 P_1^{\rm vap} + (1-x) \gamma_2 P_2^{\rm vap}}\\
\Rightarrow\  y (1-x) \gamma_2 P_2^{\rm vap} & = (1-y) x \gamma_1 P_1^{\rm vap}
\end{align*}
We could opt to solve this using a numerical tool like Solver in Excel or \inl{fsolve} in Python, or your favorite implicit solution method on your graphing calculator.  Here, I present a solution using the Newton-Raphson (NR) approach. For NR, we need to construct the function $f(x)$ that evaluates to zero when $x$ satisfies our implicit equation:  
\begin{align*}
f(x) & = y (1-x) \gamma_2(x) P_2^{\rm vap} - (1-y) x \gamma_1(x) P_1^{\rm vap},\ \ \mbox{where}\\
\gamma_1(x) & = \exp\left[A(1-x)^2\right]\ \ \mbox{and}\\
\gamma_2(x) & = \exp\left(Ax^2\right).
\end{align*}
Additionally, we need its derivative $df/dx$:
\begin{align*}
\frac{df}{dx} & = y \left[-\gamma_2+(1-x)\frac{d\gamma_2}{dx}\right] P_2^{\rm vap} - (1-y)\left[\gamma_1 + x\frac{d\gamma_1}{dx}\right] P_1^{\rm vap},\\
& = -y\gamma_2P_2^{\rm vap} \left[1-(1-x)\frac{d\ln\gamma_2}{dx}\right] - (1-y)\gamma_1P_1^{\rm vap}\left[1+x\frac{d\ln\gamma_1}{dx}\right],\\
& = -y\gamma_2P_2^{\rm vap} \left[1-2Ax(1-x)\right] - (1-y)\gamma_1P_1^{\rm vap}\left[1-2Ax(1-x)\right]\\
& = -\left[1-2Ax(1-x)\right]\left[y\gamma_2P_2^{\rm vap}+(1-y)\gamma_1P_1^{\rm vap}\right]
\end{align*}

NR says that the quantity $x - f/(df/dx)$ is a better approximation to the solution of the implicit equation than is just $x$.  So we follow an iterative strategy in which the "current" value of $x$ is updated to "new" value of $x$, which then becomes the "current" value to give as another "new" value, and so on.  When the absolute value of the correction term $f/(df/dx)$ falls below some tolerance $\epsilon$, then we stop and claim victory.

To begin, we need an initial guess for $x$.  We know that ethanoate (species 1) is much more volatile than $n$-heptane (species 2) at the temperature because it has the higher vapor pressure.  So that means that we expect that, in the absence of any azeotropes, that the liquid in equilibrium with a vapor will have a much lower mole fraction of ethanoate than that vapor does.  So we expect $x$ to be much less than \py{f'{ydew:.2f}'}. Let's begin with $x$ = \py{f'{xinit:.2f}'} and use a tolerance $\epsilon$ of \py{tu.sci_notation_as_tex(epsilon,mantissa_fmt='{:.1e}')}:

\begin{table}[h]
\begin{center}
\py{nrdf_str}
\end{center}
\end{table}

So the value of $x$ that satisfies the implicit equation is \py{f'{xdew:.4f}'}.  This is the composition of the droplet of dew that condenses from a vapor with composition $y_1$ = \py{f'{ydew:.2f}'} at \py{f'{T:.2f}'} K.  The pressure at which this condensation occurs is the dew-point pressure:
\begin{align*}
    P & = x_1\gamma_1P_1^{\rm vap} + x_2 \gamma_2 P_2^{\rm vap}\\
    & = (\py{f'{xdew:.4f}'})(\py{f'{gdew[0]:.2f}'})(\py{f'{Pv[0]:.2f}'}) + (1-\py{f'{xdew:.4f}'})(\py{f'{gdew[1]:.2f}'})(\py{f'{Pv[1]:.2f}'})\\
    & = \fbox{\py{f'{Pdew:.2f}'}\ \mbox{kPa}}.
\end{align*}

(Checking using \inl{fsolve}, I get $x_1$ = \py{f'{xdew:.5f}'} at the dew-point.)

\item[(c)]  At the azeotrope, we know $x_i = y_i$, so
\begin{align*}
    y_iP & = x_i\gamma_iP_i^{\rm vap}\\
    \Rightarrow\ P& = \gamma_iP_i^{\rm vap} = \gamma_1P_1^{\rm vap} = \gamma_2P_2^{\rm vap}\\
    \Rightarrow\ \frac{\gamma_1}{\gamma_2} &= \frac{P_2^{\rm vap}}{P_1^{\rm vap}}
\end{align*}
Note that this final equation appears implict for $x_1$ at the azeotrope.  However, because the activity coefficient model is so simple, we can solve for $x_1$ explicitly:
\begin{align*}
    \gamma_1 & = \frac{P_2^{\rm vap}}{P_1^{\rm vap}}\gamma_2\\
    e^{A(1-x_1)^2} & = \frac{P_2^{\rm vap}}{P_1^{\rm vap}}e^{Ax_1^2}\\
    A(1-x_1)^2 & = \ln\frac{P_2^{\rm vap}}{P_1^{\rm vap}} + Ax_1^2\\
    A(1-2x_1+x_1^2-x_1^2) & = \ln\frac{P_2^{\rm vap}}{P_1^{\rm vap}}\\
    1-2x_1 & = \frac{1}{A}\ln\frac{P_2^{\rm vap}}{P_1^{\rm vap}}\\
    \Rightarrow\ x_1 & = \frac{1}{2}-\frac{1}{2A}\ln\frac{P_2^{\rm vap}}{P_1^{\rm vap}}\\
    & = \frac{1}{2}-\frac{1}{2(\py{Aij})}\ln\left(\frac{\py{Pv[1]}}{\py{Pv[0]}}\right) = \fbox{\py{f'{xazeo:.4f}'}}.
\end{align*}

Peforming a bubble-point pressure calculation at this composition gives the pressure at the azeotrope:
\begin{align*}
    P & = x_1 \gamma_1 P_1^{\rm vap} + x_2 \gamma_2 P_2^{\rm vap}\\
    & = (\py{f'{xazeo:.4f}'})(\py{f'{gazeo[0]:.4f}'})(\py{f'{Pv[0]:.4f}'}) + (1-\py{f'{xazeo:.4f}'})(\py{f'{gazeo[1]:.4f}'})(\py{f'{Pv[1]:.4f}'}) = \fbox{\py{f'{Pazeo:.2f}'}~kPa}.
\end{align*}

\IfFileExists{mypxy-68549586-4.3.pdf}{
The figure below shows the full $Pxy$ diagram for this binary at \py{f'{T:.2f}'} K, with the azeotrope labeled.

\begin{figure}[h]
\begin{center}
\includegraphics[width=0.5\textwidth]{mypxy-68549586-4.3.pdf}
\end{center}
\end{figure}
}{
not showing plot yet
}

\end{enumerate}
\clearpage
\else
\ifgivespace
\clearpage
\noindent (This page left blank for extra space.)
\clearpage
\fi
\fi

\end{enumerate}

\begin{pycode}
if 'STReq' in locals():
    steamtables = STReq.to_latex()
    if len(steamtables) > 0:
        print(steamtables)

from shutil import rmtree

if len(pythonTexFC) > 0:
   # pickle it
   pkl = pickle_cache / f"pythontex-{serial}.pkl"
   pkl.write_bytes(pickle.dumps(pythonTexFC, protocol=pickle.HIGHEST_PROTOCOL))

if 'safe_mplconfig' in locals():
    rmtree(safe_mplconfig)

if 'AnsSet' in locals():
    # pickle it
    pkl = pickle_cache / f"answers-{serial}.pkl"
    pkl.write_bytes(pickle.dumps(AnsSet, protocol=pickle.HIGHEST_PROTOCOL))


\end{pycode}

\end{document}

% End of automatically generated LaTeX source file
