% Author: Cameron F. Abrams, <cfa22@drexel.edu>
%
% To use this template, create a yaml file containing a list
% of items, each of which needs the following key:value pairs:
%
%  Q: the question text
%  A: the answer (T or F)
%  text: an explanation to be included in the solutions
%  
% refer to this in as the 'config' in the exam description yaml file.  If you
% are building a series of exams in a subdirectory, prepend the name with 
% '../' 
\begin{pycode}
import numpy as np
import yaml
rng=np.random.default_rng(seed=52899270)
with open('../my-TF-questions.yaml','r') as f:
    TF=yaml.safe_load(f)
rng.shuffle(TF)
answers=[x['A'] for x in TF]
soln='52899270_3_soln.yaml'
with open(soln,'w') as f:
    yaml.dump(dict(source='true_false.tex',serial=52899270,answers=answers),f)
\end{pycode}
True/False questions. Write ``T'' for ``True'' or ``F'' for ``False'' in the blank space.
\newcommand{\fillblank}{\ul{\ \ \ \ \ \ }}
\newcommand{\tf}[1]{\item[\ifshowsolutions\textcolor{blue}{\ul{{#1}}}\else\fillblank\fi]}
\begin{pycode}
print(r'\begin{itemize}'+'\n')
for qdict in TF:
    print(r'\tf{'+qdict['A']+r'} '+qdict['Q']+'\n'+r'\ifshowsolutions\textcolor{red}{'+qdict['text']+r'}\fi'+'\n')
print(r'\end{itemize}'+'\n')
\end{pycode}
\clearpage

