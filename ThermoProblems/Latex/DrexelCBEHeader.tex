\documentclass[11pt]{article}
\usepackage{datatool}
\usepackage{graphicx}
\usepackage{amsfonts, amssymb, amsmath, latexsym, epic, eepic, url, enumitem}
\usepackage{multirow}
\usepackage{palatino}
\usepackage[mathscr]{euscript}
\usepackage{xcolor}
\usepackage[margin=1in]{geometry}
\usepackage{accents}
\usepackage{xifthen}
\usepackage{cancel}
\usepackage{listings}
\usepackage{xfp}
\usepackage[version=4]{mhchem}
\usepackage{booktabs}
\usepackage[font=small,labelfont=bf]{caption}
\usepackage{pythontex}

\newif\ifshowsolutions
\newif\ifshowanswers
\newif\ifgivespace

\newsavebox{\fmbox}
\newenvironment{fmpage}[1]
{\begin{lrbox}{\fmbox}\begin{minipage}[c]{#1}}
{\end{minipage}\end{lrbox}\fbox{\usebox{\fmbox}}}

\definecolor{Brown}{cmyk}{0,0.81,1,0.60}
\definecolor{OliveGreen}{cmyk}{0.64,0,0.95,0.40}
\definecolor{CadetBlue}{cmyk}{0.62,0.57,0.23,0}
\definecolor{lightlightgray}{gray}{0.9}
\lstdefinestyle{mybash}{language=bash,basicstyle=\ttfamily\color{white},backgroundcolor=\color{black},numbers=none,keywordstyle=\bf\color{white},showspaces=false,showstringspaces=false}
\lstdefinestyle{myc}{language=C,basicstyle=\ttfamily\color{CadetBlue},backgroundcolor=\color{lightlightgray},frame=none,numbers=none,keywordstyle=\bf\color{CadetBlue},tabsize=2,showspaces=false,showstringspaces=false}
\lstdefinestyle{mypython}{
language=Python,                             % Code langugage
basicstyle=\footnotesize\ttfamily,                   % Code font, Examples: \footnotesize, \ttfamily
keywordstyle=\color{OliveGreen},        % Keywords font ('*' = uppercase)
commentstyle=\color{gray},              % Comments font
numbers=left,                           % Line nums position
numberstyle=\tiny,                      % Line-numbers fonts
stepnumber=1,                           % Step between two line-numbers
numbersep=5pt,                          % How far are line-numbers from code
backgroundcolor=\color{lightlightgray}, % Choose background color
frame=none,                             % A frame around the code
tabsize=4,                              % Default tab size
captionpos=b,                           % Caption-position = bottom
breaklines=true,                        % Automatic line breaking?
breakatwhitespace=false,                % Automatic breaks only at whitespace?
showspaces=false,                       % Dont make spaces visible
showstringspaces=false,
showtabs=false,                         % Dont make tabls visible
numbers=none
%columns=flexible,                       % Column format
}
\newcommand{\inl}[1][]{%
  \lstinline[style=mypython,basicstyle=\ttfamily,#1]%
}

\DeclareRobustCommand{\molar}[1]{\underaccent{\bar}{#1}}
\newcommand{\parmol}[2]{\overline{#1}_{#2}}
\def\ub{\molar{U}}
\def\hb{\molar{H}}
\def\sb{\molar{S}}
\def\vb{\molar{V}}
\def\gb{\molar{G}}
\def\thetab{\molar{\theta}}
\newcommand{\dm}[1]{\Delta_{\rm mix} {#1}}
\newcommand{\deriv}[2]{\frac{d{#1}}{d{#2}}}
\newcommand{\pv}{P^{\rm vap}}
\newcommand{\pd}[2]{\frac{\partial {#1}}{\partial {#2}}}
\newcommand{\tpd}[3]{\left(\frac{\partial {#1}}{\partial {#2}}\right)_{#3}}
\newcommand{\ppd}[2]{\frac{\partial^2 {#1}}{\partial {#2}^2}}
\newcommand{\dr}[1]{\Delta_{\rm rxn} {#1}}
\newcommand{\df}[1]{\Delta_{\rm f} {#1}^\circ}
\newcommand{\hr}{\dr{H^{\circ}}}
\newcommand{\gr}{\dr{G^{\circ}}}
\newcommand{\hf}{\df{H}}
\newcommand{\gf}{\df{G}}

\newcommand{\Universityname}{\so{\sc Drexel University}}
\newcommand{\Departmentname}{Department of Chemical and Biological Engineering}
\newcommand{\Coursename}{CHE 330 -- Chemical Engineering Thermodynamics II}
\newcommand{\psheader}[4]{
    \showsolutionsfalse
    \ifthenelse{\equal{#4}{soln}}{\showsolutionstrue}{\showsolutionsfalse}
    \ifthenelse{\equal{#4}{ans}}{\showanswerstrue}{\showanswersfalse}
    \begin{centering}
        \Universityname\\
        \Departmentname\\
        \Coursename\ -- {#1} -- Prof. Abrams\\
        P\ R\ O\ B\ L\ E\ M\ \ \ S\ E\ T\ \ \ {#2}\\
        Due {#3}\\*[5mm]

        \ifshowsolutions
        \color{blue}{\bf S\ O\ L\ U\ T\ I\ O\ N\ S}\color{black}\\
        \fi
    \end{centering}
}
\newcommand{\examheader}[4]{
    \showsolutionsfalse
    \ifthenelse{\equal{#4}{soln}}{\showsolutionstrue}{\showsolutionsfalse}
    \ifthenelse{\equal{#4}{ans}}{\showanswerstrue}{\showanswersfalse}
    \begin{centering}
        {\sc D\ R\ E\ X\ E\ L\ \ \ U\ N\ I\ V\ E\ R\ S\ I\ T\ Y}\\
        Department of Chemical and Biological Engineering\\
        \Coursename\ -- {#1} -- Prof. Abrams\\
        {#2}\\
        {#3}\\*[5mm]
        \ifshowsolutions
        \color{blue}{\bf S\ O\ L\ U\ T\ I\ O\ N\ S}\color{black}\\*[1cm]
        \fi
        \end{centering}
}
\newcommand{\morespace}[1]{
\noindent More space for problem #1
\clearpage
}
\newcommand\examquestion[2]{\givespacetrue\newcommand\totalpoints{#1}\input{#2}\givespacefalse}
\newcommand\practicequestion[1]{\givespacefalse\input{#1}}
\newcommand\problem[1]{\givespacefalse\input{#1}}
\graphicspath{{C:/Users/cfa22/OneDrive - Drexel University/Teaching/Thermodynamics II (CHE 330)/Problems/For Exams/}{C:/Users/cfa22/OneDrive - Drexel University/Teaching/Thermodynamics II (CHE 330)/Problems/For Problem Sets/}}
\makeatletter
\def\input@path{{C:/Users/cfa22/OneDrive - Drexel University/Teaching/Thermodynamics II (CHE 330)/Problems/For Exams/}{C:/Users/cfa22/OneDrive - Drexel University/Teaching/Thermodynamics II (CHE 330)/Problems/For Problem Sets/}}
\makeatother